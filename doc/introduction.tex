\paragraph{}
Natural language understanding is one of the fields of natural language processing (NLP), which is under ongoing research. This work tries to bring improvements into this field, targeting expert (scientific) domains which employ specific language. Appropriate examples of such domains can be found among the 173 Stackexchange's question and answer platforms. The topics covered range from computer science to biology and chemistry to personal finance. The individual domains require to target specific aspects of the users' language, which we demonstrate on the Stackoverflow page. However, general principles shown in this work can be applied to other domains as well.

\paragraph{}
The future usage of this work is to improve tasks such as information retrieval or code generation, which is beyond the scope of this work. In the case of the firstly mentioned, the vector representations of questions obtained by proposed neural networks can be used to search for the same questions already asked and answered. Therefore, it would be possible to get the correct answer immediately after describing the problem. Our goal is to outperform standard keyword-based approaches. Furthermore, our work can also be used for automatic duplicate question detection on question and answer platforms.

\paragraph{}
An approach in this work is to use marked duplicate questions from the Stackoverflow to train a neural network to classify whether two questions are duplicates (i.e., they are describing the same problem). Given this task, hidden layers of the neural network are forced to generate vector representations of the input questions, which are the desired outcome. An essential aspect of this work is to utilize the information contained in code snippets present in the questions.

\paragraph{}
This bachelor thesis is structured as follows. In the first chapters, basic concepts and techniques of neural networks and natural language processing are briefly described. Later in the subsequent chapters, results and realization of the given targets are going to be presented in detail.